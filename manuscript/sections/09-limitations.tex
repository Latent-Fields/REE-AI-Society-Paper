
Section 9. Limitations and Future Directions

This paper is theoretical and synthetic rather than empirical. It integrates concepts from artificial intelligence architecture, cognitive science, ethics, and clinical observation to identify a class of risks associated with asymmetric cognitive amplification. Many of the phenomena discussed—such as epistemic drift, reinforcement loops, and authority diffusion—are difficult to operationalise experimentally and often unfold over extended periods within relational and social systems.

The Reflective–Ethical Engine (REE) is not evaluated here as an implemented system. Its role in this paper is illustrative, demonstrating that ethical stability can, in principle, be internalised architecturally rather than imposed solely through external governance or behavioural constraints. The present analysis does not claim that REE is complete, optimal, or empirically validated.

Psychiatric concepts are used functionally rather than diagnostically. The intention is not to medicalise artificial intelligence use or its users, but to draw on established accounts of cognitive instability to illuminate structural failure modes. Future work should clarify the boundaries between transient interactional effects, subclinical destabilisation, and clinically significant psychopathology.

Constraint-based cognitive architectures raise substantial implementation challenges. These include representing long-horizon coherence, integrating self–other modelling, maintaining uncertainty without paralysis, and balancing revisability with decisiveness. Whether such constraints can be realised efficiently, robustly, and at scale remains an open empirical question.

Future research directions include longitudinal studies of human–machine interaction, experimental comparisons between assistant-style and constraint-based systems, and clinical research examining artificial intelligence use as a contributing factor in cognitive destabilisation under real-world conditions. Such work will be essential for testing the claims advanced here and for informing the design of cognitively sustainable artificial intelligence systems.
