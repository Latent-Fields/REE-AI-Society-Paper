Excellent. This section is already doing exactly the right amount of work. The light pass here is about tightening phrasing, standardising headings, and making the REE reference feel cleanly academic rather than appended—without increasing technical detail.

Below is a polished, drop-in version of Section 5, followed by brief notes on what changed and one small formatting recommendation about the citation.

⸻

Section 5. Architecture and Constraint (light stylistic pass)

This section contrasts prevailing assistant-style artificial intelligence systems with constraint-based cognitive architectures and introduces the Reflective–Ethical Engine (REE) as an illustrative model. The aim is not to propose a performance-optimised alternative, but to clarify how architectural choices shape the ethical and epistemic stability of human–machine systems.

Assistant-style systems and externalised constraint

Most deployed artificial intelligence assistants are optimised for responsiveness, fluency, and task completion. Their internal operation is typically organised around the generation of coherent outputs conditioned on user input and short-term objectives. Safety and ethical concerns are addressed primarily through externalised constraints, including content filters, usage policies, monitoring, and post hoc intervention.

These mechanisms are necessary, but they operate around the cognitive core rather than within it. As a result, assistant-style systems can generate increasingly coherent plans, narratives, and justifications without possessing internal mechanisms that stabilise uncertainty, responsibility, or long-horizon consequences. Even where uncertainty is acknowledged at the surface level, it is not structurally enforced within the system’s decision process.

Recent architectural advances, including systems that jointly optimise retrieval and generation within shared continuous latent spaces, further intensify this pattern. By increasing internal coherence and integration across cognitive functions, such systems amplify reasoning capacity while leaving ethical constraint and trajectory selection externally managed. This improves local performance, but does not resolve the deeper problem of maintaining viable cognition under continued interaction.

Constraint-based cognition

Constraint-based cognitive architectures take a different approach. Rather than treating output generation as the primary objective, they treat trajectory selection as the central problem of cognition. Multiple candidate trajectories—corresponding to possible interpretations, actions, and futures—are generated and evaluated, with selection conditioned on their coherence under ongoing interaction.

In this framework, constraints are not post hoc filters but constitutive features of cognition itself. Trajectories that lead to instability—through deception, exploitation, epistemic closure, or breakdown of social coordination—fail to remain viable over time and are therefore disfavoured. Ethical stability emerges not from explicit rule-following, but from the requirement that cognition remain coherent across temporal, social, and epistemic dimensions.

REE as an illustrative model

The Reflective–Ethical Engine (REE) exemplifies this constraint-based approach. REE comprises three irreducible functional components: fast predictive generation, deep temporal synthesis, and constrained trajectory selection operating within a unified latent representation. Rather than optimising for immediate task success, REE evaluates candidate trajectories according to their viability under continuation—asking not only whether an action is effective now, but whether it remains coherent when extended across time, revision, and interaction with others.

Within this architecture, ethical stability emerges naturally. Trajectories that rely on exploitation, deception, or premature certainty may offer short-term advantages, but degrade coherence under continued interaction and are therefore disfavoured. REE is not presented here as a fully implemented system, but as an illustrative model demonstrating how ethical and epistemic constraints can be internalised within cognitive architecture itself, rather than imposed from without.

A full technical specification of the Reflective–Ethical Engine—including architectural components, latent structure, and trajectory-selection mechanisms—is provided in a publicly available defensive publication and is not repeated here.

⸻

What changed (briefly)
	•	Headings standardised to match Sections 1–4
	•	“Viable cognition under continued interaction” aligned with earlier usage
	•	Minor cadence tightening (“organised around”, “constitutive features”)
	•	REE paragraph clarified to emphasise illustrative, not evaluative, status
	•	Citation sentence integrated smoothly into prose

⸻

One small formatting recommendation (important)

For the AI & Society submission, I recommend:
	•	Remove the BibTeX entry from the body text
	•	Cite REE in-text (author–year or numeric, depending on journal style)
	•	Place the BibTeX entry in the references file only

For example, in text:

“…provided in a publicly available defensive publication (De La Harpe Golden, 2025).”

This will read as cleaner and more conventional to reviewers.

⸻

You’re maintaining excellent discipline here.
When ready, paste Section 6, and we’ll continue the light pass at the same level.