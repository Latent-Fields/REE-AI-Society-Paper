

Conclusion

This paper has argued that a central risk of contemporary artificial intelligence lies not primarily in autonomy or intent, but in asymmetric cognitive amplification: the enhancement of human cognitive capacities without a commensurate strengthening of ethical and epistemic constraints. By reframing artificial intelligence risk as a problem of constraint failure, the analysis helps explain why subtle but consequential phenomena—such as epistemic override, narrative lock-in, and agency capture—can arise through ordinary, well-intentioned use.

In response, the paper has proposed constraint-based cognitive architectures as a principled direction for mitigation. In such architectures, exemplified here by the Reflective–Ethical Engine, ethical stability is not imposed as an external rule or policy layer, but emerges as a necessary condition for cognition to remain viable over time under uncertainty. From this perspective, ethical behaviour is inseparable from the structural conditions that allow cognition—human or artificial—to remain revisable, socially embedded, and responsive to consequence.

The implication is that meaningful artificial intelligence safety cannot be achieved through governance, control, or behavioural compliance alone. It must also attend to the internal cognitive structures that shape how coherence, confidence, and action are generated and sustained. As artificial intelligence systems increasingly function as cognitive amplifiers rather than autonomous agents, the question of safety becomes inseparable from the question of what forms of cognition these systems make easy, rewarding, or inevitable. Addressing that question is central to whether artificial intelligence functions as a stabilising or destabilising influence within human cognitive and social life.

