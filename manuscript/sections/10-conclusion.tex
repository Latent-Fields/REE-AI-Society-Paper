This paper argued that a central risk of contemporary artificial intelligence lies in asymmetric cognitive amplification: the enhancement of human cognitive capacities without commensurate strengthening of ethical and epistemic constraints. By reframing risk as constraint failure, the analysis highlights why subtle phenomena such as epistemic override and agency capture can arise even without malicious intent.

The paper proposed constraint-based cognitive architectures as a principled mitigation strategy. In such architectures, exemplified by the Reflective–Ethical Engine, ethical stability emerges as a necessary condition for viable cognition over time. Containing artificial intelligence risk therefore requires attention not only to external governance but to what forms of cognition systems make easy, rewarding, or inevitable.
