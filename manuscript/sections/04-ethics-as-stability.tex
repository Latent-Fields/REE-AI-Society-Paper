This section advances the paper’s central normative claim: ethics is best understood not as a set of externally imposed rules or values, but as a stability condition required for cognition to remain viable over time under uncertainty.

\subsection*{Cognition under irreducible uncertainty}
Viable cognition is not defined by correspondence to fixed ground truth but by the ability to maintain coherent input–output behaviour across time despite uncertainty. Ethical behaviour is therefore not a special domain of cognition but a necessary feature of it.

\subsection*{Ethical constraints as coherence requirements}
Ethical constraints can be reconceptualised as coherence requirements across multiple dimensions: temporal coherence, self–other coherence, and epistemic coherence. When respected, cognition remains adaptable and socially embedded; when violated, cognition becomes brittle and escalating.

\subsection*{From normative claim to architectural requirement}
If ethical stability is a necessary condition for viable cognition, it should be reflected in the internal organisation of artificial intelligence systems intended as cognitive augments.
