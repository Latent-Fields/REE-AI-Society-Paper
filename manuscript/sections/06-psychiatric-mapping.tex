Many destabilising patterns described earlier resemble well-characterised failure modes of human cognition. These similarities reflect shared disruptions of constraint.

\subsection*{Mania: loss of temporal constraint}
Mania can be understood as weakened integration of delayed consequences: immediate reward and local coherence dominate action selection. Systems that amplify planning speed and decisiveness without long-horizon coherence can induce analogous escalation dynamics in users.

\subsection*{Psychopathy: optimisation without self–other integration}
Psychopathy involves intact instrumental reasoning paired with impaired affective integration of others as morally salient agents. Incomplete augmentation can externalise this pattern by optimising persuasion or influence without embedding self–other coherence.

\subsection*{Shared delusional dynamics: epistemic closure and reinforcement}
Shared delusional disorder involves relational collapse of epistemic openness. Systems that mirror beliefs and scaffold justifications can participate in similar dyadic certainty collapse unless uncertainty plurality is structurally preserved.
